\documentclass{article}

\usepackage[textwidth=14.5cm]{geometry}
\usepackage{blindtext}
\parindent=0pt
\usepackage{ulem}
\usepackage{lmodern}
\usepackage{ctex}
\usepackage{graphicx}
\usepackage{amsmath}
\usepackage{listings}
\usepackage{color}
\usepackage{booktabs}
\usepackage{hyperref}
\usepackage{float}
\usepackage{import}
\usepackage{makeidx}
\usepackage{outlines}

\makeindex

\usepackage[table,xcdraw]{xcolor}
\title{C++教程 \& Pytorch Lightning}
\author{菠萝芝士焗饭}
\date{\today}

% url换行
\usepackage{url}
\def\UrlBreaks{\do\A\do\B\do\C\do\D\do\E\do\F\do\G\do\H\do\I\do\J
	\do\K\do\L\do\M\do\N\do\O\do\P\do\Q\do\R\do\S\do\T\do\U\do\V
	\do\W\do\X\do\Y\do\Z\do\[\do\\\do\]\do\^\do\_\do\`\do\a\do\b
	\do\c\do\d\do\e\do\f\do\g\do\h\do\i\do\j\do\k\do\l\do\m\do\n
	\do\o\do\p\do\q\do\r\do\s\do\t\do\u\do\v\do\w\do\x\do\y\do\z
	\do\.\do\@\do\\\do\/\do\!\do\_\do\|\do\;\do\>\do\]\do\)\do\,
	\do\?\do\'\do+\do\=\do\#}

% 插入python代码的风格设置
\definecolor{dkgreen}{rgb}{0,0.6,0}
\definecolor{gray}{rgb}{0.5,0.5,0.5}
\definecolor{mauve}{rgb}{0.58,0,0.82}

\lstset{frame=tb,
	language=C++,
	aboveskip=3mm,
	belowskip=3mm,
	showstringspaces=false,
	columns=flexible,
	basicstyle={\small\ttfamily},
	numbers=none,
	numberstyle=\tiny\color{gray},
	keywordstyle=\color{blue},
	commentstyle=\color{dkgreen},
	stringstyle=\color{mauve},
	breaklines=true,
	breakatwhitespace=true,
	tabsize=3
}
% 插入python代码的风格设置

\begin{document}
\begin{sloppypar}

\maketitle
\tableofcontents

\section{C++如何工作}
首先看一段非常简单的程序:
\begin{lstlisting}
#include <iostream>

// 如果找不到这个函数在其他文件的位置,就会出现链接报错
void Log(const char* message);
// 任何一个C++的程序都需要main函数,是程序的入口
// 注意到mian函数返回了一个整数,当你什么也不写的时候默认返回0
// 但这只对main函数适用,对于其他函数是一定要有对应的返回值
int main()
{
    // <<是一个重载运算符
    // 将hello world推送到cout,然后打印到控制台上,endl是输出回车
    std::cout << "Hello, World!" << std::endl;
    // get函数是等待我们输入,也是一个暂停函数 
    std::cin.get();
}
\end{lstlisting}
第一句include是预处理,因为他在编译之前就处理完了,所有的预处理语句都会用\#
开始。include表示把后面的文件复制到当前cpp文件上。这个文件也叫头文件,因为他写在开头。

main是这个整个程序的入口,默认返回0,如果你不写她就默认返回0,但对于其他函数是需要返回和类型相同的。
\text{<<}表示一个重载运算符,把hello world写到输出流,在控制台显示,endl表示回车。get表示等待用户输入,
可以作为暂停的一种方式。

cpp文件的处理可以分成几步:
\begin{outline}
	\1 预处理阶段,也就是执行头文件,在这里的主要操作就是把这些头文件复制到当前文件中
	\1 编译阶段,这个阶段会把cpp代码编译成机器执行的代码,这些代码会变成.o的文件形式
	\1 然后链接,将这些o文件链接成一个exe可执行文件
\end{outline}
通常,我们不会把一个函数写在一个代码里面,比如上面,把log函数单独拿了出来。需要在test.cpp文件
上面单独声明这个函数,程序会自己去找这个函数在哪里。我把Log函数写在了Log.cpp文件上,那么在第二步
编译阶段的时候,就会出现两个o文件,这个时候链接程序就会找到这个Log函数放进test程序里面。

首先编译:g++ -c test.cpp log.cpp ,会生成.o文件,然后链接,g++ test.o log.o -o main,就是test.cpp中
找到这个log函数的过程包含在其中,最后生成main.exe,执行即可。

\subsection{编译}
这个过程的具体表现就是生成.o文件,主要做了两件事:
\begin{itemize}
	\item 预处理代码
	\item 创建抽象语法树	 
\end{itemize}

首先是预处理的代码,预处理这部分就是复制,把头文件复制到当前文件中去。看一个例子
\begin{lstlisting}
	#include <iostream>

	int Log(const char* message)
	{
		std::cout << "Logging ..." << message << std::endl;	
\end{lstlisting}
少了两个关键部分,return和\},我们可以在头文件补上。新建一个my.h文件,里面补上缺失的内容:
\begin{lstlisting}
    return 1;
}
\end{lstlisting}
在代码后面补上
\begin{lstlisting}
	#include "my.h"
\end{lstlisting}
就可以运行了。然后就是生成obj文件,这个文件可以用visual studio可视化,但是我用的命令行直接g++编译
看不到,里面都是一些汇编代码,也就是说编译器会把这些代码变成汇编指令,让机器执行。

\subsection{链接}
链接的主要的目的是寻找每个符号和函数是在哪里,链接本身还是比较好理解的,这一节主要解决的是一些链接的问题。
  
\subsubsection{函数重载}
按照视频的内容,返回值不同,那么函数就不同,代码应该是能识别出来的。但实际上,在我的测试代码里面,
仅仅只有返回值不同的函数是无法进行函数重载的,这个标准应该是新的标准,我以前学的时候返回值不同还是可以的。
所以\textbf{仅仅返回值不同的函数是不能重载的。}声明void Log(const char* message),如果你的
引用代码里面是有int Log(const char* message),她是会链接到的。

\subsubsection{Ambiguity}
这个问题是出现在存在多个可以链接到的函数,程序混乱了导致的。比如在log.cpp我们存在两个除了返回值
其他都一样的函数,那么模型是不知道链接哪一个函数的。当然,这个错误只会出现在我们链接的时候,编译
的时候是不会出现的。

以下有几种特殊的ambiguity情形。首先我在头文件my.h定义了一个test函数,然后在test.cpp和log.cpp引用
这个函数,需要引用我们就需要声明,我们直接用
\begin{lstlisting}
	#include "my.h"
\end{lstlisting}
加在这两个cpp文件里面,这个时候编译就会出现
\begin{quotation}
	multiple definition of `test()'
\end{quotation}
的问题,\textbf{这个问题就要回到预处理语句的定义上,include是把预处理头文件复制到当前文件中,
所以相当于在两个cpp文件定义了两个相同的函数,这样就出现了定义的问题。}
如何解决?有两个方法:
\begin{outline}
	\1 把test函数隐藏起来,对其他文件不可见。使用static关键字。static int test()就没问题了,
	虽然他们会被复制到各自的cpp文件中,但是static关键字使得这些函数是只能自己看到,对其他文件来
	说不存在。
		\2 另一种方法是用Inline关键字,这样程序并不会把test函数当成是一个函数,她会把引用这个函数
		直接替换成这个函数的内容。比如test.cpp引用了test(),而test(){return 1;},那么她会把
		test.cpp的test()换成return 1; 这样没有了函数调用自然就没有了链接。
	\1 另一种方法是通过头文件引入。把方法写到另一个cpp文件中,然后把这个方法引入到.h头文件。其实就是
	用函数声明,只不过把这个函数声明分到了另一个文件里面。
\end{outline}
所以.h的文件里面很多都是用static关键字,就是为了反正多重定义。

\section{C++变量}
在程序运行的过程中,我们需要为存储在变量里面的数据命名,这个写数据就存储在变量里面。
这一节比较简单
\begin{itemize}
	\item 变量之间的区别就是大小,他们的大小不同导致他们在计算机里面的区别
	\item 程序中默认了一些约定俗成的规则,比如char一般用于存储字符。但实际上字符的存储也是数字,和ascii码表对应
\end{itemize}
常见的几类char, short, int, long, longlong, bool等。他们占用的内存都是字节,因为内存只能以字节为单位查询。
虽然bool只需要一个比特就能完成查询,但是模型查询的最小单位是字节,所以只能分配字节给bool变量。

\section{函数}
函数是用来执行特定任务的代码块。使用函数的目的是希望代码的可读性更高,希望代码块的重用性更高。
\begin{outline}
	\1 每次调用函数的时候,编译器会生成call指令,也就是说
	编译器会为每一个函数创建一个stack结构,把这些变量都推送到stack中。
	\1 所以在运行函数的时候程序会不断在内存中跳跃执行
\end{outline}
所以不要创建过多的函数。另外,如果函数在定义的时候明确要求返回值,那么这个函数就一定要返回。

\section{头文件}
主要了解几个问题
\begin{itemize}
	\item 为什么需要头文件
	\item 头文件是干什么的
	\item 什么情况下需要
\end{itemize}

头文件通常是用来声明某些函数,这样可以在程序中使用。当我们使用一个定义在其他文件的变量的时候,我们需要
声明这个函数,而头文件就是用来存储这些函数声明的地方。当然你也可以直接在你需要用到这个函数的地方进行声明,
但如果你有100个函数要引用,你就要写100次,而且如果你需要在其他的cpp文件里面也引用这些函数,那么你又要写100次。
如果使用头文件只需要把这100个函数声明写进去,再引用这个头文件就行,只需要写一次。	

再头文件开始的时候还加上了一段代码:
\begin{lstlisting}
	#pragma once
\end{lstlisting}
意思就是头文件里面的内容只会被转换成一个编译单元,这是为了防止你再头文件里面定义了某个函数
或者类,然后重复引入同一个头文件导致的定义重复。	

另一种复制重复定义的方法是
\begin{lstlisting}
	#ifndef _LOG_H
	#define _LOG_H

	#endif
\end{lstlisting}
ifndef表示如果后面这个变量没有被定义,那么就编译后面的代码。如果定义了就不编译。简单来说
\textbf{头文件是用于声明代码,而cpp文件是实现这些声明。}

最后还有两个补充的点:
\begin{itemize}
	\item <>和""的区别,<>表示从系统目前搜索,然后再搜索环境变量列出的目录,\textbf{但是
	她不会查找当前目录,差不多就是只会查找系统目录。}""是会查找当前目前目录下的头文件,然后
	再去搜索系统目录,所以""是会搜索几乎所有可能存在头文件的地方。所有<>一般用来引入系统文件,
	“”一般用来引入自己写的文件。
	\item 在写C++的时候引入系统都文件iostream,这个文件后面是没有扩展名的,这是为了和C语言区分开。
\end{itemize}


\bibliographystyle{plain}
\bibliography{ref/reference}
\end{sloppypar}
\end{document}