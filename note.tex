\documentclass{article}

\usepackage[textwidth=14.5cm]{geometry}
\usepackage{blindtext}
\parindent=0pt
\usepackage{ulem}
\usepackage{lmodern}
\usepackage{ctex}
\usepackage{graphicx}
\usepackage{amsmath}
\usepackage{listings}        
\usepackage{color}
\usepackage{booktabs}
\usepackage{hyperref}
\usepackage{float}
\usepackage{import}
\usepackage{makeidx}
\usepackage{outlines}
\usepackage[backend=bibtex]{biblatex}

\makeindex

\usepackage[table,xcdraw]{xcolor}
\title{C++教程 \& Pytorch Lightning}
\author{菠萝芝士焗饭}
\date{\today}

% url换行
\usepackage{url}
\def\UrlBreaks{\do\A\do\B\do\C\do\D\do\E\do\F\do\G\do\H\do\I\do\J
	\do\K\do\L\do\M\do\N\do\O\do\P\do\Q\do\R\do\S\do\T\do\U\do\V
	\do\W\do\X\do\Y\do\Z\do\[\do\\\do\]\do\^\do\_\do\`\do\a\do\b
	\do\c\do\d\do\e\do\f\do\g\do\h\do\i\do\j\do\k\do\l\do\m\do\n
	\do\o\do\p\do\q\do\r\do\s\do\t\do\u\do\v\do\w\do\x\do\y\do\z
	\do\.\do\@\do\\\do\/\do\!\do\_\do\|\do\;\do\>\do\]\do\)\do\,
	\do\?\do\'\do+\do\=\do\#}

% 插入python代码的风格设置
\definecolor{dkgreen}{rgb}{0,0.6,0}
\definecolor{gray}{rgb}{0.5,0.5,0.5}
\definecolor{mauve}{rgb}{0.58,0,0.82}

\lstset{frame=tb,
	language=C++,
	aboveskip=3mm,
	belowskip=3mm,
	showstringspaces=false,
	columns=flexible,
	basicstyle={\small\ttfamily},
	numbers=none,
	numberstyle=\tiny\color{gray},
	keywordstyle=\color{blue},
	commentstyle=\color{dkgreen},
	stringstyle=\color{mauve},
	breaklines=true,
	breakatwhitespace=true,
	tabsize=3
}
% 插入python代码的风格设置

\begin{document}
\begin{sloppypar}

\maketitle
\tableofcontents

\section{前言}
一些笔记,来自视频\url{https://www.bilibili.com/video/BV1uy4y167h2/?spm_id_from=333.788&vd_source=a059a118f33728f79abd79e02f8f72d4}
只是一些简单记录,代码类的还是得上手实践,就算完完整整的全部记下来,最后也很容易忘记。就像学数学一样
不做题不巩固,就不知道这些知识点要怎么用。所以简单过一遍,然后上项目,项目遇到了什么不懂的再
回来补充。

\section{C++基础教程}

\subsection{C++如何工作}
首先看一段非常简单的程序:
\begin{lstlisting}
#include <iostream>

// 如果找不到这个函数在其他文件的位置,就会出现链接报错
void Log(const char* message);
// 任何一个C++的程序都需要main函数,是程序的入口
// 注意到mian函数返回了一个整数,当你什么也不写的时候默认返回0
// 但这只对main函数适用,对于其他函数是一定要有对应的返回值
int main()
{
    // <<是一个重载运算符
    // 将hello world推送到cout,然后打印到控制台上,endl是输出回车
    std::cout << "Hello, World!" << std::endl;
    // get函数是等待我们输入,也是一个暂停函数 
    std::cin.get();
}
\end{lstlisting}
第一句include是预处理,因为他在编译之前就处理完了,所有的预处理语句都会用\#
开始。include表示把后面的文件复制到当前cpp文件上。这个文件也叫头文件,因为他写在开头。

main是这个整个程序的入口,默认返回0,如果你不写她就默认返回0,但对于其他函数是需要返回和类型相同的。
\text{<<}表示一个重载运算符,把hello world写到输出流,在控制台显示,endl表示回车。get表示等待用户输入,
可以作为暂停的一种方式。

cpp文件的处理可以分成几步:
\begin{outline}
	\1 预处理阶段,也就是执行头文件,在这里的主要操作就是把这些头文件复制到当前文件中
	\1 编译阶段,这个阶段会把cpp代码编译成机器执行的代码,这些代码会变成.o的文件形式
	\1 然后链接,将这些o文件链接成一个exe可执行文件
\end{outline}
通常,我们不会把一个函数写在一个代码里面,比如上面,把log函数单独拿了出来。需要在test.cpp文件
上面单独声明这个函数,程序会自己去找这个函数在哪里。我把Log函数写在了Log.cpp文件上,那么在第二步
编译阶段的时候,就会出现两个o文件,这个时候链接程序就会找到这个Log函数放进test程序里面。

首先编译:g++ -c test.cpp log.cpp ,会生成.o文件,然后链接,g++ test.o log.o -o main,就是test.cpp中
找到这个log函数的过程包含在其中,最后生成main.exe,执行即可。

\subsubsection{编译}
这个过程的具体表现就是生成.o文件,主要做了两件事:
\begin{itemize}
	\item 预处理代码
	\item 创建抽象语法树	 
\end{itemize}

首先是预处理的代码,预处理这部分就是复制,把头文件复制到当前文件中去。看一个例子
\begin{lstlisting}
	#include <iostream>

	int Log(const char* message)
	{
		std::cout << "Logging ..." << message << std::endl;	
\end{lstlisting}
少了两个关键部分,return和\},我们可以在头文件补上。新建一个my.h文件,里面补上缺失的内容:
\begin{lstlisting}
    return 1;
}
\end{lstlisting}
在代码后面补上
\begin{lstlisting}
	#include "my.h"
\end{lstlisting}
就可以运行了。然后就是生成obj文件,这个文件可以用visual studio可视化,但是我用的命令行直接g++编译
看不到,里面都是一些汇编代码,也就是说编译器会把这些代码变成汇编指令,让机器执行。

\subsubsection{链接}
链接的主要的目的是寻找每个符号和函数是在哪里,链接本身还是比较好理解的,这一节主要解决的是一些链接的问题。
  
\subsubsection{函数重载}
按照视频的内容,返回值不同,那么函数就不同,代码应该是能识别出来的。但实际上,在我的测试代码里面,
仅仅只有返回值不同的函数是无法进行函数重载的,这个标准应该是新的标准,我以前学的时候返回值不同还是可以的。
所以\textbf{仅仅返回值不同的函数是不能重载的。}声明void Log(const char* message),如果你的
引用代码里面是有int Log(const char* message),她是会链接到的。

\subsubsection{Ambiguity}
这个问题是出现在存在多个可以链接到的函数,程序混乱了导致的。比如在log.cpp我们存在两个除了返回值
其他都一样的函数,那么模型是不知道链接哪一个函数的。当然,这个错误只会出现在我们链接的时候,编译
的时候是不会出现的。

以下有几种特殊的ambiguity情形。首先我在头文件my.h定义了一个test函数,然后在test.cpp和log.cpp引用
这个函数,需要引用我们就需要声明,我们直接用
\begin{lstlisting}
	#include "my.h"
\end{lstlisting}
加在这两个cpp文件里面,这个时候编译就会出现
\begin{quotation}
	multiple definition of `test()'
\end{quotation}
的问题,\textbf{这个问题就要回到预处理语句的定义上,include是把预处理头文件复制到当前文件中,
所以相当于在两个cpp文件定义了两个相同的函数,这样就出现了定义的问题。}
如何解决?有两个方法:
\begin{outline}
	\1 把test函数隐藏起来,对其他文件不可见。使用static关键字。static int test()就没问题了,
	虽然他们会被复制到各自的cpp文件中,但是static关键字使得这些函数是只能自己看到,对其他文件来
	说不存在。
		\2 另一种方法是用Inline关键字,这样程序并不会把test函数当成是一个函数,她会把引用这个函数
		直接替换成这个函数的内容。比如test.cpp引用了test(),而test(){return 1;},那么她会把
		test.cpp的test()换成return 1; 这样没有了函数调用自然就没有了链接。
	\1 另一种方法是通过头文件引入。把方法写到另一个cpp文件中,然后把这个方法引入到.h头文件。其实就是
	用函数声明,只不过把这个函数声明分到了另一个文件里面。
\end{outline}
所以.h的文件里面很多都是用static关键字,就是为了反正多重定义。

\subsection{C++变量}
在程序运行的过程中,我们需要为存储在变量里面的数据命名,这个写数据就存储在变量里面。
这一节比较简单
\begin{itemize}
	\item 变量之间的区别就是大小,他们的大小不同导致他们在计算机里面的区别
	\item 程序中默认了一些约定俗成的规则,比如char一般用于存储字符。但实际上字符的存储也是数字,和ascii码表对应
\end{itemize}
常见的几类char, short, int, long, longlong, bool等。他们占用的内存都是字节,因为内存只能以字节为单位查询。
虽然bool只需要一个比特就能完成查询,但是模型查询的最小单位是字节,所以只能分配字节给bool变量。

\subsection{函数}
函数是用来执行特定任务的代码块。使用函数的目的是希望代码的可读性更高,希望代码块的重用性更高。
\begin{outline}
	\1 每次调用函数的时候,编译器会生成call指令,也就是说
	编译器会为每一个函数创建一个stack结构,把这些变量都推送到stack中。
	\1 所以在运行函数的时候程序会不断在内存中跳跃执行
\end{outline}
所以不要创建过多的函数。另外,如果函数在定义的时候明确要求返回值,那么这个函数就一定要返回。

\subsection{头文件}
主要了解几个问题
\begin{itemize}
	\item 为什么需要头文件
	\item 头文件是干什么的
	\item 什么情况下需要
\end{itemize}

头文件通常是用来声明某些函数,这样可以在程序中使用。当我们使用一个定义在其他文件的变量的时候,我们需要
声明这个函数,而头文件就是用来存储这些函数声明的地方。当然你也可以直接在你需要用到这个函数的地方进行声明,
但如果你有100个函数要引用,你就要写100次,而且如果你需要在其他的cpp文件里面也引用这些函数,那么你又要写100次。
如果使用头文件只需要把这100个函数声明写进去,再引用这个头文件就行,只需要写一次。	

再头文件开始的时候还加上了一段代码:
\begin{lstlisting}
	#pragma once
\end{lstlisting}
意思就是头文件里面的内容只会被转换成一个编译单元,这是为了防止你再头文件里面定义了某个函数
或者类,然后重复引入同一个头文件导致的定义重复。	

另一种复制重复定义的方法是
\begin{lstlisting}
	#ifndef _LOG_H
	#define _LOG_H

	#endif
\end{lstlisting}
ifndef表示如果后面这个变量没有被定义,那么就编译后面的代码。如果定义了就不编译。简单来说
\textbf{头文件是用于声明代码,而cpp文件是实现这些声明。}

最后还有两个补充的点:
\begin{itemize}
	\item <>和""的区别,<>表示从系统目前搜索,然后再搜索环境变量列出的目录,\textbf{但是
	她不会查找当前目录,差不多就是只会查找系统目录。}""是会查找当前目前目录下的头文件,然后
	再去搜索系统目录,所以""是会搜索几乎所有可能存在头文件的地方。所有<>一般用来引入系统文件,
	“”一般用来引入自己写的文件。
	\item 在写C++的时候引入系统都文件iostream,这个文件后面是没有扩展名的,这是为了和C语言区分开。
\end{itemize}

\subsection{条件和分支}
 在写代码的时候,我们需要根据结果或者根据某些条件判断我们应该执行哪些语句。但实际上在代码中
 添加分支会减慢代码的速度。这部分比较简单,主要讨论一些char *和数组的内容。

 首先有几个关键Insight:
 \begin{outline}
	\1 数组是多个元素的组合,他们的地址的连续的
	\1 指针的类型为整形int,永远为四个字节,定义的时候float * q表示q指向一个float类型的数据
	\1 数组的本质就是首个元素的地址
 \end{outline}

在本章的示例代码中,定义字符串是通过:
\begin{lstlisting}
	const char * s = "Neural Radiance Field";
\end{lstlisting}
以前学的时候不需要const是没问题,可能是标准不同了,右边的NeRF字符串是一个常量,类型是const char *,
左边和右边的类型对不上,所以报错了。输出的时候直接针对这个地址输出即可:
\begin{lstlisting}
	std::cout << s << std::endl;
\end{lstlisting}
如果定义成
\begin{lstlisting}
	char arr[] = "Neural Radiance Field";
\end{lstlisting}
也是一样,arr就是字符串的首地址,输出地址对比可以看到是一样的:
\begin{lstlisting}
	char * p = arr;
    std::cout << static_cast<void *>(p) << static_cast<void *>(&arr[0]) << std::endl;
\end{lstlisting}

\subsection{For\&while}
这两个都是用来表示循环,for的格式
\begin{lstlisting}
	for(int i = 0; i < 5; i++)
	{

	}
\end{lstlisting}
定义变量,变量条件,变量自增。如果想到达成无限循环,可以在条件部分补上true,或者直接不写。
如果想要达成等差数列,或者等比数列,你可以直接在变量自增部分改成i+=2, i+=3, i*=3等。

while也是类似:
\begin{lstlisting}
	while(i < 5)
	{
		i ++;
	}
\end{lstlisting}
每次都会判断是否小于5,while之后的括号一定要写上i自增或者其他改变i的语句,否则就会一直循环。
但实际上,同样,要达成无限循环也只需要在条件写true,或者你对i这个变量不处理就行;等差数列和等比数列也只需要
对i这个变量的自增处理一下。所以while和for其实是可以互换的。

除此之外还有一个do while的循环,do while和while其实区别不大,主要在于do while是一定要走一次的,
while可以一次都不走。

\subsection{控制流}
主要介绍三个控制流语句:
\begin{outline}
	\1 continue
	\1 break
	\1 return
\end{outline}
continue用于跳过当前循环直接进入到下一个循环。Break推出整个循环。return直接结束。

\subsection{指针}
对于程序来说,最重要的部分就是内存,当写下程序运行的时候,程序会被写入到内存中,控制语句,
循环语句本质上也是内存之间的不断跳跃。\textbf{每一个变量第一会存储在内存中,为了找到这些变量,
我们需要为每一个内存编号,这个编号会由另外一个变量存储,这个存储编号的变量就是指针。}

定义一个指针
\begin{lstlisting}
	void* ptr = nullptr;
\end{lstlisting}
指针无论前面定义什么类型,都是一个整形的指针,前面定义的类型只是表明了她需要指向什么样类型的数据。nullptr表示
目前这是一个空指针。现在定义一个变量,用这个指针指向它:
\begin{lstlisting}
	int var = 8;
	ptr = &var;
\end{lstlisting}

\*可以通过指针访问指向的内存数据,\&可以获得当前变量的地址。

如果我们想要一些内存存储,但是又不知道内容的时候,就采用分配的方法。
\begin{lstlisting}
    char* buffer = new char[8];
    memset(buffer, 0, 8);
\end{lstlisting}
new char是分配一个空间为8字节的char类型内存空间,返回指向这个空间的指针。memset是填充这个内存,
这里表示用8个0填充。除此之外,还有指针的指针等。

\subsection{引用}
引用通常就是地址的伪装,引用和指针很大的区别在于指针可以先定义,然后设置他们称为一个空指针;
但是引用不可能创建一个空的引用,必须要引用已经存在的变量。 

假设一个已知的变量a,可以通过
\begin{lstlisting}
	int& ref = a;
    ref = 90;
\end{lstlisting}
创建一个a的引用ref,ref就是a的别名,对于ref的改动其实就是对a的改动,之所以说引用没有内存是因为
她和a共用地址。其实通常并不是会创建一个新的变量指向a,通常引用是在函数中使用,希望能直接在函数中修改变量的值。  

定义一个自增函数
\begin{lstlisting}
void Increment(int value)
{
    value ++;
}
\end{lstlisting}
直接把a传入进去,是不会改变a的值。解决这个问题有两个方法,可以利用指针传递,参数改成输入指针:
\begin{lstlisting}
void Increment(int* value)
{
    (*value) ++;
}
\end{lstlisting}
需要加括号,不加括号相当于是取出下一个地址的内容,然后不做处理。这样相当于获得内存的地址,直接对地址进行修改。
另一种方法就是再方法的参数加上\&引用,传入一个引用。

\subsection{类}
这部分开始进入面向对象编程。假设一个场景,我们需要写游戏,编写一个玩家的类信息。
当然我们可以直接把玩家的所有信息都写进main函数里面,位置xy,速度velocity等等。
如果定义另外一个人,又需要重新定义一次位置,速度等等。简单定义一个类:
\begin{lstlisting}
class Player
{
public:
    // 公有变量,意味着可以在任何地方访问
    int x, y;
    int speed;

    void Move(int xa, int ya)
    {
        x += xa * speed;
        y += ya * speed;
    }

};
\end{lstlisting}
C++的类如果默认设置就是私有的,除了自己谁也访问不了,所有通常会写一些公有的访问函数。

\subsection{类和结构体的对比}
主要是对比结构体和类的区别,说实话学了类之后很少写结构体了,类的功能比结构体强大很大,
完全可以替代结构体的功能。

事实上这两者的区别不大,从技术上讲,两者的区别在于结构体的内容都是公有的,但是类是私有的。
当然了结构体也可以定义private使得变量函数私有。

\subsection{static}
static这个关键字在之前就已经见过,用于限制方法的可见范围。通常写在头文件的都是static。
如果static写在类的内部,那么说明这个变量是共享的。

在类外的定义:
\begin{lstlisting}
	static int static_variable = 5;
\end{lstlisting}
这个变量只会在当前的编译单元见到。之前这个做法是用来解决链接冲突的问题。A,B两个cpp文件
都引用了头文件.h的一个log函数,预处理代码会先把头文件的内容都拷贝到A,B文件中,所以这
两个文件实际上是定义了两个相同函数,链接的时候编译器就会疑惑怎么会有两个相同的函数。
为了解决这个问题,我们把头文件的函数写成static,这样辅助到cpp文件的函数也是static,
他对其他cpp文件不可见,这样就只有一个函数可以链接。

至于为什么需要static,这和类中为什么需要private变量一样。

\subsection{类和结构体内部的static}
上一节的static是定义在方法,变量前面,也就是类别外面的。这节的static是定义在
类内部。

static成员只能在类内声明不能初始化,要初始化static成员必须在类外初始化。比如
创建Player类别,在类内定义两个变量x, y。static变量初始化一定要在所有代码之前,
初始化是在编译器初始化。也就是放在所有程序的外面。

有两点需要注意:
\begin{outline}
	\1 静态变量的初始化要在所有代码之前,所有一般在头文件写好然后引用到cpp。
	\1 类内静态函数只能访问静态变量。和python,java类似,每一个类内函数
	都会传入一个类的实例,C++里面的this指针,默认传入,python传入self对象。静态方法
	没有实例也就是没有this指针。
\end{outline}

\subsection{局部静态}
在定义变量的时候,我们需要考虑变量的两个属性,生命周期和作用域。通常限制变量
生命周期是变量定义的位置,限制作用域通常由特定关键字,比如static, inline等,
或者定义的位置决定。

假设函数add,我们希望作用于一个全局变量上,可以如此定义:
\begin{lstlisting}
int sum = 0;
void sum_f()
{
    sum ++;
}
\end{lstlisting}
但如果我们不想要让其他函数对次变量修改编辑,我们就可以把这个变量编程静态局部变量:
\begin{lstlisting}
void sum_f()
{
    static int sum = 0;
    sum ++;
    LOG(sum);
}
\end{lstlisting}
此时sum变量就变成局部可见了,仅仅只有sum函数可见了。用处还是蛮多的,可以用来
debug,测试该方法跑了几回;也可以用这种方法来实现一个类的功能,因为他可以隐藏变量,
相当于定义一个全局变量。

总结一下static的用法:
\begin{outline}
	\1 static在类外定义,在方法或变量前面,该方法和变量仅仅对于该文件可见
	\1 static在类内定义,该方法相当于类的单例变量,实例通过.访问,类通过::访问
	\1 static在方法内定义,相当于把全局变量的可见范围缩减到该方法内部。
\end{outline}

\subsection{枚举}
枚举实际上就是一个数值组合,实际上是一种清洁代码的方法。比如
假设日志的级别,我们会给0,1,2这些数字赋予一些人为的约定,比如0是警告等,
这个时候如果用枚举就可以简化我们的书写。
\begin{lstlisting}
	enum levels{
    A, B, C, D, E, F, G, H, I, J, K
};
\end{lstlisting}
这样就创建一个枚举类,这个类里面定义的A,B,...都是static变量,完全可以通过levels::A访问,
或者直接就A也能访问,这样能让代码更直观。可读性更强。当然也可以直接创建一个实例,但是
这个实例只能填ABC...那几个给好的数字。

\subsection{构造函数}
java也有这东西。构造函数是用来初始化变量的,在定义一个类的时候,我们有时候需要一些
特定的操作去初始化函数,比如python中的init函数。这个时候如果把他们全部写在变量一起,
代码看起来就会很混乱,所以构造函数就可以增加模型的可读性。

以类名为函数名,不需要准备返回值类型,就可以定义一个构造函数。
\begin{lstlisting}
	Player(int x, int y)
    {
        
    }
\end{lstlisting}
在实例化一个类的时候就会自动调用。

\subsection{析构函数}
这个函数是在删除实例的时候调用,也就是在该类释放内存的时候调用。

\subsection{继承}
类之间的继承是C++的最强大特性之一。我们可以通过继承定义一个互相关联的类的层次结构,
这些不同的子类都包含了一个共同的祖先。
先定义一个父类:
\begin{lstlisting}
class Player
{
public:
    // 公有变量,意味着可以在任何地方访问
    static int x, y;
    int speed;

    Player(int x, int y)
    {

    }

    void Move(int xa, int ya)
    {
        x += xa * speed;
        y += ya * speed;
    }

    static void s_print()
    {
        LOG(x);
        LOG(y);
    }

};
\end{lstlisting}
同时定义一个子类继承他:
\begin{lstlisting}
class Good : public Player{
	public:
		using Player::Player;
	};
\end{lstlisting}
Good这个子类会拥有这个父类Public的所有变量和方法。using这条语句是C++11的标准,
可以用这种方法把父类的构造函数继承下来。所以在Good类中不用初始化了。注意:
\begin{outline}
	\1 如果一个function要求参数是Entity类,那么可以输入Entity及其子类。但是
	如果定义了子类,就不能传父类,因为父类有可能没有子类方法。比如定义了Player类作为参数
	类型,那么可以传Good,但是用Good类作为参数类型就不能传Player。这种其实就多态,
	自适应各种不同类型。
	\1 继承父类之后重载函数实际上是重新定义函数,而不是重载某一个父类函数。比如
	父类有aabb(int i, int j)和aabb()两个函数,子类继承了下来,子类希望重新覆盖
	aabb函数,于是子类自己定义了一个aabb函数,那么父类的aabb(int i, int j)也会被隐藏起来。
\end{outline}

\subsection{虚函数}
视频中列举了一种情况,现在存在两个类,他们是继承关系:
\begin{lstlisting}
class Entity{
public:
     void printName(){
        LOG("Entity");
    }
};

class player : public Entity{
    public:
      void printName(){
        LOG("Player");
     }
};

\end{lstlisting}
编译运行以下语句:
\begin{outline}
void testName(Entity* entity){
    entity->printName();
}

int main()
{
    Entity* entity = new Entity();
    Entity* ply = new player();
    testName(entity);
    testName(ply);
}
\end{outline}
会输出两个Entity,我们原本是创建了一个plyer类但是却被识别成Entity的类,使用了
父类的函数。\textbf{因为通常在声明函数的时候,我们的方法通常是在类的内部起作用,
这个也很容易理解,因为这两个指针实际上是经过了强转的,模型是不知道他们的来历的,
只能知道他们的类型是什么,除此之外就一模一样了。所以这个时候我们就要想个办法让模型能意识到,
当前的指针的来自player的。}

虚函数就是解决这个问题,使用虚函数模型会创建一个v表,这个表会记录当前虚函数的映射,
也就是被哪个函数复写了。每次运行都需要查表,当然这就出现了额外的性能损失。

\subsection{接口}
接口函数比虚函数更极端,他会强制子类去实现这个虚函数。他不需要实现任何东西,只需要给函数
参数就行。

但是继承的子类需要强制实现,类似于给子类定义了一个标准,按照这个标准去声明。

\subsection{可见性}
这部分比较重要。这里的可见性一般是指类成员的可见性。

之前在提到类的时候,类成员定义的时候不加修饰默认是私有。

\begin{outline}
	\1 private:私有变量只能当前类,或者友元类能访问。
    \1 protected:相比于private,他的可见性更强,因为他的子类是能访问到的。
\end{outline}
所以为什么要设置可见性?

首先代码里面纯用public是一个比较糟糕的做法,因为对于开发者和程序来说,你需要保证
你代码的安全性,比如在管理数据库的时候你需要写好API提供给你的程序员调用,如果全部开放
很容易就出现删库跑路或者数据泄露的问题。

\subsection{数组}
数组和指针息息相关,数组就是指针的基础,可以通过对于指针的操作对数组直接操作。

创建数组有两种形式:
\begin{itemize}
	\item 直接调用a[n]创建数组。这种方式是在栈上创建,到达第一个花括号就会自动释放。
	\item 调用new创建,那么会创建在堆上,不会自动释放,除非关闭了程序。
\end{itemize}
这两者最大的区别就是生存周期,其次还有一定性能的区别,new创建的数组实际上是用指针去接受
了这个数组的第一个元素的地址,然后再根据这个地址寻找其他变量。这是一种间接寻址,
而在栈上创建是直接寻址。

视频后面讲了一堆,感觉没有什么记录的必要,主要需要记录的是栈数组初始化的时候:
\begin{lstlisting}
class Entity{
    public:
        const int size = 5;
        int example[size];
};
\end{lstlisting}
这种初始化是错误的,需要在size前面加上static,因为编译器不能确定example初始化前,你这个size
和array哪一个先创建,很可能是同时创建的。如果你直接在main函数,或者在其他函数这样创建是没有问题的,
\begin{lstlisting}
int main()
{   
    const int size = 5;
    int example[size];
}
\end{lstlisting}
最后提了一嘴STL。

\subsection{字符串}
另一种数组,只不过用来表达字符串而已。
\begin{lstlisting}
	const char* name = "new array";
    std::cout << name << std::endl;
\end{lstlisting}
定义了const之后就不能对模型进行改变,定义了name并且赋值之后,这个字符串的最后
会补上一个0,表示字符串的终结,这也是为什么直接输出name能完整的输出字符串。

如果用数组定义,那就需要显示的补上0,否则就会出现乱码:
\begin{lstlisting}
    const char name1[10] = {'n', 'e', 'w', ' ', 'a', 'r', 'r', 'a', 'y', 'a'};
    std::cout << name1 << std::endl;
\end{lstlisting}
这样就会出现乱码。

另外STL也有string,可以直接用,不够C++对编译器优化了很多问题已经不存在了。

\subsection{const}
const就像一种承诺,定义的时候就承诺定义的变量我们就不会改变,但是既然是承诺那也能违背,
所以其实是可以通过某种方法去绕过这个变量的。当我们定义一个变量a = 5,可以随意改变这个变量的值,
但如果前面加了const,那就不能修改这个变量的值了。

首先是const和指针搭配使用:
\begin{itemize}
	\item const int\*和int const\*是一个意思,可以改变指针的指向,但不能改变指针指向的内容,
	也就是能使得指针b = \&a某个变量,但是不能\*b = a。
	\item int\* const能改变指针的内容,但是不能改变指针的指向。
	\item 放在方法的后面,\textbf{这种做法只能在类中使用,int getX() const{return x},
	意味着你不能修改类的变量。}通常在需要print,查看类信息的时候使用。
\end{itemize}

极端点的情况:
\begin{lstlisting}
const int* const getX() const
	{
		return &x;
	}
\end{lstlisting}
这个函数有三个const,当然是在类里面定义的函数,返回的指针不能修改指向,不能修改内容,并且这个
函数也不能修改类变量。

突然发现设计一个好的语言真的很烦,随便一个改动就要考虑各种各样的情况。有一个需要调用
这个getX类内函数的普通函数
\begin{lstlisting}
void print_Get(Entity& entity)
{
   entity.getX();
}
\end{lstlisting}
这样调用是没问题的,getX的const是限制类内函数和外面的环境没有关系。但是如果我在参数加上
const,那么getX就不能把后面的const留下了,也就是说这样是错误的:
\begin{lstlisting}
class Entity{
    public:
        int x = 4;

        static const int size = 5;
        int example[size];

        const int* const getX()
        {
            return &x;
        }
};

void print_Get(const Entity& entity)
{
   entity.getX();
}
\end{lstlisting}
因为输入print\_Get函数的时候你已经要求了entity不变,但是如果getX不加const,是可以改变的,
那么这个const就失去了意义。所以这个getX就需要加const。所以经常会看到两个版本的getX
就是在这个原因。

但如果你确实是想做一些标记,希望能在const的情况下继续修改,那么就需要mutable关键字,这样你的变量就可以
修改了。总之const就类似一些强行施加的规定,保证代码的安全性。

\subsection{成员初始化列表}
主要是对类成员初始化所用。最常见的一种初始化方式就是构造函数了,定义不同的构造函数
可以使用不同的方式进行初始化。

另一种初始化就是成员列表初始化,在构造函数后面加上冒号按照顺序写好:
\begin{lstlisting}
class Player{
    public:
        int x, y;
        int velocity;
        
        // 列表初始化需要按照声明变量的顺序来写,顺序不能乱
        Player(): x(0), y(0), velocity(0)
        {

        }

        Player(int x, int y, int velocity): x(x), y(y), velocity(velocity)
        {
            
        }
};
\end{lstlisting}
所以为什么我们需要使用这个,成员列表初始化和一般的初始化方法其实结果都是一样的,完全可以互相替代,
但是如果使用成员列表初始化,就可以对代码进行简化,当你有很多变量的时候,
一般的初始化会使得构造函数很难看出是在做什么,

但如果你用成员列表初始化,构造函数就会很清晰。初次之外,使用成员初始化还能增加程序的性能。
假设类内构造函数初始化如下图所示:
\begin{lstlisting}
	int x, y;
	int velocity;
	Entity* entity;
	
	// 列表初始化需要按照声明变量的顺序来写,顺序不能乱
	// Player(): x(0), y(0), velocity(0), entity()
	// {
	 
	// }
	Player()
	{
		entity = new Entity(8);
	}

\end{lstlisting}
那么Entity会创建两次,第一次是定义的时候,第二次是构造函数声明的时候,如果用构造成员列表,
那么只会构建一次:
\begin{lstlisting}
	Player(): x(0), y(0), velocity(0), entity(Entity(7))
	{
	 
	}
\end{lstlisting}
\textbf{最后的结论就是尽量使用成员列表初始化。}

\subsection{三元操作符}
三元操作符实际上是一种简化代码的形式,if语句的一种语法糖。 正常if的写法:
\begin{lstlisting}
	if (level < LEVEL){
        level = LEVEL;
    }
\end{lstlisting}
用语法糖可以一行写完:
\begin{lstlisting}
	level = level < LEVEL ? LEVEL : level;
\end{lstlisting}

有两个优点:
\begin{itemize}
	\item 代码更简洁
	\item 速度更快,因为他不会创建一个临时的字符串作为比较结果
\end{itemize}

\subsection{创建以及初始化类}
首先是栈和堆的区别,内存主要也是分成两部分,堆和栈,栈的生命周期是和他在哪里声明相关,
如果在函数中声明,那么函数结束的时候就会把这些变量都弹出来删除。堆不一样,堆内的对象
不会自动释放,而是会等待你做出释放的命令,否则他会一直停留到结束。对于类来说,默认初始化
就是在栈上创建,new关键字创建就是在堆上创建。

\subsection{new关键字}
\textbf{如果希望用到C++,一般都要关注于内存,性能,并行计算和优化问题。}Java,python这类
语言带有内存清理的机制,C++没有,需要自己控制内存。

使用new初始化类,是一种非常常见的操作了。new实际上是一个操作符,她不仅仅分配了空间,她还调用
了类的构造函数。\textbf{new实际上是调用了c语言的malloc分配函数}
\begin{lstlisting}
	Entity* e = new Entity();
	Entity* e = (Entity*)malloc(sizeof(Entity));
\end{lstlisting}
上面两行的区别在于第二行只是分配了空间,而第一行不仅分配了空间,还调用了构造函数。
\textbf{在调用new操作符的时候,需要手动清理内存空间,因为在堆上的变量是不会自动释放的。
所以如果使用了new,最后要接上delete。}

\subsection{隐式转换和explicit关键字}
首先作者举了一个类的例子,对于有构造函数
\begin{lstlisting}
	Entity(int a)
	{
		cout << "create " << a << endl;
	}

\end{lstlisting}
的类,我们可以直接对她进行赋值,Entity a = 23;。这里会自动做一次隐式变换,
\textbf{隐式变换只能做一次,如果需要两次变换的会报错。}
比如Entity a = "67"就有问题,她需要两次变换,首先变成string,然后变成int,再变成
entity。这种代码的书写方式可读性不高,所以如果想禁止这种方式可以使用explicit,
只需要再构造函数之前加上explcit。

隐式变换在书写代码的过程中处处都有,比如对于一个函数参数为double类型的函数,如果调用
时候输入1,那么她会把1变成double类型,这就是一种隐式变换。

\subsection{运算符重载}
\begin{outline}
	\1 首先什么是运算符:运算符是一种符号,可以用来代替函数进行执行,常见的数字运算符
	加减乘除,new,delete或者问号都是运算符。
\end{outline}

看一个简单的重载例子,定义一个类
\begin{lstlisting}
class Vector_two{
    public:
        int x, y;
        Vector_two(int a, int b):x(a), y(b) {}
};

Vector_two speed(1, 3);
Vector_two position(2, 4);

\end{lstlisting}

如果想要把这两个类对应的元素相加,可以写一个类内的函数调用,另一种比较方便的
写法是重载+运算符。
\begin{lstlisting}
	Vector_two operator+(const Vector_two& b) const
    {
        return Vector_two(x + b.x, y + b.y);
    }

    Vector_two operator*(const Vector_two& b) const
    {
        return Vector_two(x * b.x, y * b.y);
    }
\end{lstlisting}
重载加号和乘号的运算符。

这是在类内的重载,我们也可以重载<<
\begin{lstlisting}
ostream& operator<<(ostream& stream, const Vector_two& other)
{
    stream << other.x << ", " << other.y;
    return stream;
}
\end{lstlisting}

\subsection{对象生存期}
\begin{itemize}
	\item 如何理解栈上的创建的对象?
	\item 如何写出高效的代码?
\end{itemize}
栈可以认为是一种数据结构,这种数据结构的特点就是,她会在作用域结束之后删除,
如果你在一个函数里面不用new方法创建了一个类,然后返回指向这个类的指针,
那么在函数结束之后这个类就会被释放,返回的指针指向了一个空区域。

\subsection{智能指针}
这算是一个新东西了,在使用关键字new创建一个新的变量的时候,我们需要配套使用delete删除
这个内存,这样相对来说比较麻烦,而智能指针的对这一过程简化。引入memory头文件,
\begin{lstlisting}
int main()
	{
		{
			// std::unique_ptr<Entity> entity(new Entity());
			// 这种方式最安全
			std::unique_ptr<Entity> entity = std::make_unique<Entity>(); 
		}
	}
\end{lstlisting}
这样就创建一个entity对象,跳出这个作用域那么智能指针会自动删除。但是智能指针是不能
复制的,也就是说当智能指针指向了这个内存后不能有其他指针指向了,因为如果智能指针删除
了当前变量,那么另一个指针就没有效果了。

shared\_ptr是另一种指针,她的释放内存方式和python的差不多,都是计算内存指向的指针数目,如果
是0那么就释放。shared\_ptr解决了unique\_ptr不能复制的问题。weak\_ptr是shared\_ptr的
一种,她不会增加shared\_ptr的引用次数。

\subsection{C++复制与拷贝函数}
在python中复制还分深度拷贝还浅度拷贝,其实这里说的就是这两个概念。
创建两个变量:
\begin{lstlisting}
	int a = 3;
    int b = a;
\end{lstlisting}
这种就相当于深度复制,a和b相当于两个不同的变量,只不过数值相同而已。如果我用指针复制:
\begin{lstlisting}
   int* a = new int();
   int* b = a;
\end{lstlisting}
那a和b指针相当于共享内存了。

看一个简单的例子:
\begin{lstlisting}
	class String
{
    private:
        char* m_Buffer;
        unsigned int m_Size;
    public:
        String(const char* p)
        {
           this->m_Size = strlen(p);
           this->m_Buffer = new char[this->m_Size];
           memcpy(m_Buffer, p, this->m_Size);
        }
        
        // 声明为友元重载,可以访问私有变量
        friend std::ostream& operator<<(std::ostream& os, const String&);
};

std::ostream& operator<<(std::ostream& os, const String& string)
{ 
    os << string.m_Buffer;
    return os;
}
\end{lstlisting}
在这个类中,我们声明了一个字符串,一个友元重载,因为我们需要访问私有变量。在运行输出的时候
由于没有终止符,会出现一些错乱的输出,我们只需要在m\_Size改成m\_Size + 1就行,memcpy
多余的字符会用0填充。

但是如果我们创建另外一个变量,这个变量string赋值,那就会出现问题:
\begin{lstlisting}
    String string = "new code";
    String second = string;
    std::cout << string << std::endl;
\end{lstlisting}
因为这个时候复制的变量是指针,其中一个指针释放了内存另一个又要释放,完了找不到内存就崩溃了。
这种只是浅拷贝,要解决这个问题,就需要构造一个拷贝构造函数。
\begin{lstlisting}
	String(const String& other): m_Size(other.m_Size)
	{
	   this->m_Size = strlen(other.m_Buffer);
	   this->m_Buffer = new char[this->m_Size + 1];
	   memcpy(m_Buffer, other.m_Buffer, this->m_Size + 1);
	}
\end{lstlisting}
相当于重新分配了一个内存给这个变量,这个时候
\begin{lstlisting}
	String string = "new code";
    String second = string;
\end{lstlisting}
就相当于是两个不同的变量了。

另外一个点就是,形参也是会重新复制一份,所以如果不需要复制,那就用引用就行,这样
就不会复制,而是直接把指针传过去。\textbf{尽可能使用const去传递参数,没必要到处去复制,可以
在程序内部决定要不要复制。
}

\subsection{箭头运算符}
我感觉是不是顺序有问题,这个内容应该就在前面就讲。

总的来说就是指针用箭头。这节不打算记录了,efficients C++里面不太建议这样写。

最后作者还补了一个语法糖,简单看了一下:
\begin{lstlisting}
	class Vector3
	{
		public:
			float x, y, z;
	};
	int offset = (long long)(&((Vector3*)nullptr)->z);	
\end{lstlisting}
offset就是偏移量。首先nullpter是空指针,强转成Vector*就是初始化了一个指针。然后->得到
类内元素,\&取得当前元素的地址,再强转成long long。作者是转成int,但是我这出问题,long long
是可以隐式转int的。nullpter就是存储在0地址中,所以xyz是从0开始分配。

emmm这种代码我也不会写,就放着吧。

\subsection{动态数组}
也就是vector。一般的数组在规定了她的大小之后就不能再增加元素个数了。
但是vector是可以不断增加元素个数的。for循环可以遍历:
\begin{lstlisting}
	for(Vertex& x : vertices)
	{

	}
\end{lstlisting}
注意还是尽量使用引用,避免出现复制。

在vector的使用中,有两个部分是可以进行优化的:
\begin{outline}
	\1 vector.push\_back添加元素的时候,是现在main函数内创建,然后再移动进vector里面,
	在移动的过程中就相当于是复制了。这里复制了一次。
	\1 vector是动态数组,每一次增加都需要扩充容量,每一次扩充容量就相当于把原来的数组的复制一遍。
\end{outline}
解决方法都很直接
\begin{lstlisting}
	std::vector<Vertex> vectices;
	vectices.reserve(3);
\end{lstlisting}
直接将容量变成3,这样就不用动态扩充了。

第一个问题用emplace\_back,她并不会创建变量然后推进vector中,而是会
把创建的参数丢进vector里面,然后再vector里面创建。所以尽量还是用
emplace\_back创建吧。

\subsection{处理多返回值}
处理方式有很多种
\begin{itemize}
	\item 可以用数组或者string来接收这些返回值然后返回,包括各种数据结构了。
	\item 也可以定义一个static函数,然后引用接收参数
\end{itemize}

\subsection{templates}
模板,就是设计一套规则,让机器帮你写代码。比如需要输出不同类型的变量,
对于一个print函数可能要写很多次,对于不同的变量都需要进行重载。
\begin{lstlisting}
template<typename T> void print(T value)
{
    std::cout << value << std::endl;
}
\end{lstlisting}
类型并没有指定,等待人为指定或者自动的隐式变换指定。一开始第一次编译碰见这个函数的
时候并不会定义出来,知道编译到调用这个函数的位置才会创建,也就是只有调用这个函数的
时候才会正式创建这个函数。
\begin{lstlisting}
	print(5);
    print("cahjks");
\end{lstlisting}
当然也可以自己指定。
\begin{lstlisting}
	print<int>(5);
    print<String>("cahjks");
\end{lstlisting}

如果在类上使用模板呢?
\begin{lstlisting}
template<int N>
class Array
{
    private:
        int m_Array[N];
    public:
        int GetSize() const
        {
            return N;
        }
};
\end{lstlisting}
在声明的时候并不会报错,编译器会根据使用情况进行编译。类定义的变量类型可变可以
通过模板定义。前面加两个模板就行:
\begin{lstlisting}
	template<typename T, int N>
\end{lstlisting}

\subsection{栈和堆的比较}
当我们请求内存分配的时候,栈和堆的内存分配方式是不一样的,

栈的分配速度很快,栈的分配只需要移动栈指针即可,分配四个字节的内存,只需要
将栈指针移动四个单位即可。所以栈的内存都紧挨着的。

对于堆的分配一般是用new关键字,new关键字会调用malloc分配内存。当启动这个程序的时候,os
会分配一些内存给你,程序会去维护一个空闲列表的数据结构,里面记录了哪些内存是空闲的,哪些不是空闲的。
当使用malloc分配内存的时候,他会寻找这个空闲列表,找到一些空闲的块,分配给程序。\textbf{
	当空闲列表中内存不够的时候,就需要向os请求,这个过程是很麻烦的。
}

这两种分配方法最大的区别就是在于分配方法的速度,在栈上分配的速度非常块,只需要一条cpu的指令,
就可以快速分配。堆的分配会慢一些。所以\textbf{尽量在栈上分配,除非不能或者需要更长的生命周期。}

\subsection{宏}
一开始的时候就提到过宏,比如define,if都是,是在预处理阶段处理的语句。
\begin{lstlisting}
	#define WAIT std::cin.get()
\end{lstlisting}
注意到这里没有加;,那么你在调用的时候就需要写WAIT;,因为他是替换这行。如果
define WAIT std::cin.get()写了;号,那么WAIT就不用写了。就像前面include把\{
复制进来一样。

define就是把WAIT定义成后面那句话,在预处理的时候,他会把WAIT语句替换成后面那句话,
和inline做一样的事情。\textbf{这种方式比较蠢,不推荐这样做,因为这样会导致可读性不强。}

在一些特定的情况,宏还是有用的,比如日志的输出,在debug阶段是需要输出各种日志,但是
在运行阶段我们就不希望输出了,这个时候可以如下定义:
\begin{lstlisting}
#ifdef PR_DEBUG
#define LOG(x) std::cout << x << std::endl;
#else
#define LOG(x)
#endif
\end{lstlisting}
如果是debug模型,把PR\_DEBUG定义出来,这样就会输出日志,否则就不输出。
\textbf{宏必须是要在同一行,所以如果一行写不下,可以用\\跳到下一行写。}

\subsection{auto关键字}
auto让模型自动推断变量的类型。emmmm说实话我一般不会经常用auto,
除非是接收函数返回的时候,因为这会严重影响可读性。

\subsection{静态数组array}
和动态数组vector相对应的,他一出生就决定了他的大小,不能改变size了。
\begin{lstlisting}
	std::array<int, 5> data_array;
\end{lstlisting}
这样定义的静态数组有有个问题:不知道他的个数,没办法传递参数到函数里面,因为传递函数
需要把这个静态数组的完整类型写出来。这个时候可以使用模板
\begin{lstlisting}
	template<typename T> void print_array(const T& value) 
{
    for(int i = 0; i < value.size(); i++)
    {
        LOG(value[i]);
    }
}
\end{lstlisting}
我原本想着用auto让模型自行推断,但是不行。相比直接创建的数组,
std::array数据结构有边界检查。\textbf{应该尽量用std::array替代C语言的数组:}
\begin{itemize}
	\item 首先他提供了一系列的安全检查
	\item 其次并没有声明性能损失,还能记录数组大小。
\end{itemize}

\subsection{函数指针}
获得函数指针之后可以通过指针调用这个函数
\begin{lstlisting}
	auto function_pointer = hello_world;
    function_pointer();
\end{lstlisting}
function\_pointer的类型是void(*function\_pointer),function\_pointer只是一个名字,
可以随意取。

定义一个函数指针有两种方法
\begin{outline}
	\1 一种是直接定义,    void (*fun\_name) (int);直接定义一个fun\_name的函数指针,
    fun\_name = hello\_world;将这个函数hello\_world赋值给函数指针。
	\1 第二种是先把函数指针类型定义出来typedef void (*fun\_name) (int);再根据这个类型
	把变量定义出来fun\_name fun = hello\_world;第二种方式对后续使用更友好。
	\1 第三种就是直接auto定义了。
\end{outline}
函数指针可以让函数进行参数的传递
\begin{lstlisting}
	void print_array(const T& array, void (*prt_array)(int))
{
    for(int i = 0; i < array.size(); i++)
    {
        prt_array(array[i]);
    }
}
\end{lstlisting}
直接把函数名称传递进去就行。

\subsection{命名空间}
为什么不用using namespace std?这里说的是作者为什么不用,我经常用其实。

作者不爱用这玩意的原因:
\begin{outline}
	\1 显示的写出这些命名空间,可以知道是使用哪一个库的内容。比如使用std就能知道是再C++标准库使用的。
	\1 可能出现命名空间的冲突,如果两个命名空间里面都有同名称的函数,很容易就冲突了
\end{outline}

为什么要使用命名空间?这是下一节的内容。
\begin{itemize}
	\item 简单来说,就是为了解决在大工程中容易出现命名重复的问题。
	\item 命名空间是在限定作用域下进行。
\end{itemize}

\subsection{线程}
目前我们所完成的代码都是单线程的,一次只能做一条指令。但实际上多线程在日常中很常见,
比如在制作游戏的时候,我们通常需要等待用户输入指令,但是如果主线程等待用户输入指令,那么
整个游戏就会停下来,所以我们需要其他线程来接受用户指令,主线程运行。

main就是一个主线程,定义一个函数
\begin{lstlisting}
void DoWork()
{
    while(!is_finishing)
    {
        std::cout << "working" << std::endl;
    }
}
\end{lstlisting}
在主函数中,我们定义一个线程去运行这个函数
\begin{lstlisting}
	std::thread worker(DoWork);
\end{lstlisting}
这样我们就定义一个线程去运行这个函数,但是只是定义还没开始
\begin{lstlisting}
	std::cin.get();
    is_finishing = true;
\end{lstlisting}
表示输入任一按键推出子线程。
\begin{lstlisting}
	worker.join();
\end{lstlisting}
开始运行线程,并且主线程会等待子线程运行完毕。还有另一个detach,不过这个是主线程不等待子线程
运行完毕。

\subsection{多维数组}
多维数组是数组的数组,把一个数组里面的元素换成数组即可。其实就是创建了
很多很多个数组,然后把指向这些数组的指针收集再一起,就变成了二维指针。
多维数组就相当于多层指针的意思。
\begin{lstlisting}
	int** a2d = new int*[50];
\end{lstlisting}
a2d[0]是一个指针,指向的是个数组。上面那行代码我们只是初始化了一系列指针,并没有实质分配空间,
需要用for循环挨个挨个分配:
\begin{lstlisting}
	for(int i = 0; i < 50; i++)
    {
        a2d[i] = new int[50];
    }
\end{lstlisting}
删除的时候不能直接用delete删除,如果直接delete[][] a2d,你只是删除了这一堆指针,
而指针指向的内存并没有删除,删除也是需要循环删除,否则会造成内存泄漏。

\textbf{二维数组如果这样初始化是会存在内存不连续的问题,因为每一个指针之间的内存不是连续的,
导致运行效率的问题。up提到了个cache miss,意思就是每一次系统会把当前访问内存的一些邻居内存
的内容也收进cache里面,因为系统认为相邻的内存的访问时空频率会相近。如果内存不连续,那么这样的假设
不成立,会造成频繁的cache miss。}

二维数组不是连续的,那么一维数组是连续的吧?用一维度数组去模拟二维数组就行了。这个比较简单,
常见操作了。

\subsection{排序}
用STL了基本上就是。最简单的用法
\begin{lstlisting}
	std::vector<int> values = {3, 5, 4, 2, 1};
    std::sort(values.begin(), values.end(), std::greater<int>());
    for(int value : values)
    {
        std::cout << value << std::endl;
    }
\end{lstlisting}
第二行排序,按照降序排列。当然也可以自定义,这比较容易。后面再看很多代码的时候
大部分都是自己实现的,因为不同排序在不同情况下效率不同。

\subsection{Union}
定义多个变量占用同一个内存,大小为最大的那个变量。定义一个结构体
\begin{lstlisting}
	struct test
{
   union
   {
     struct
     {
        float a, b, c, d;
     };

     struct 
     {
        Vector v1, v2;
     };
     
     
   };
   
};
\end{lstlisting}
如果修改c和d的值,会发现vector v2的xy的值也会随之修改。这个Union想要达成的目的
是想设置一个变量能够存储多个不同类型的变量,不用定义多个类型。

\subsection{虚析构函数}
这个问题和之前的虚函数一样,都是解决编译器无法识别内存实际存储变量类型的问题。
\begin{lstlisting}
	class Base
{
    public:
        Base() {std::cout << "Base" << std::endl;}
        ~Base() {std::cout << "Delete Base" << std::endl;}
};

class Derived : public Base
{
    public:
        Derived() {std::cout << "Derived" << std::endl;}
        ~Derived() {std::cout << "Delete Derived" << std::endl;}
};
\end{lstlisting}
定义一个类,另一个继承。在初始化的时候,显示调用的构造函数,模型自然是知道
这个类就是Derived,但是删除的时候模型并不知道这个内存里面的内容是哪一个类,
只能通过类型判断,这个时候就需要使用虚函数了。\textbf{如果析构函数不对,
会出现内存泄漏。}

只需要在析构函数上上virtual即可,他会创建一个新表记录。
\subsection{类型转换}
主要还是显示类型转换,定义double变量,如果强行赋值给int变量,编译器会默认使用隐式转换,
因为这样会有精度丢失,当然可以指定显示转换static\_cast等。这些cast会自带编译器的一些
安全检查,会有一定的速度损失。

作者后面给了一堆例子,没看太懂,暂不记录。cast的类别有四类
\begin{outline}
	\1 static\_cast: 针对的是精度算是较小的转换,比如浮点和整数,不能进行指针的转换
	\1 reinterpret\_cast: 用于不同类型指针的强转,常用于继承类之间的转换
	\1 const\_cast: 去除const属性
	\1 dynamic\_cast: 不带安全检查,仅仅只在运行的时候进行检查,如果失败返回null。这个可以用来
	判断一个类是否是其子类,当然也有instance可以用。
\end{outline}

\subsection{Safe}
C++编程中的安全是指降低崩溃,内存泄漏,非法访问等问题。\textbf{随着C++11的到来,
更应该转向智能指针而不是原始指针。这是作者说的。}
\begin{itemize}
	\item 内存泄漏问题:在一个堆上分配内存,如果不人为进行删除,那么这个内存将一直存在。
	如果存在循环的话这个内存就会一直分配。
	\item 内存清理的所有权问题:由谁来清理管理内存。
\end{itemize}
这集up估计是被评论喷了回一波,感觉没什么内容在里面,过。

\subsection{预编译头文件}
主要解决的问题就是编译速度过慢。每一次在头文件引入vector,string这些头文件的时候,
都需要重新对vector,string这些头文件代入的代码进行重新编译,速度会慢很多。

预编译头文件就是引入他们的二进制格式的文件,这样会加快编译。

\subsection{dynamic\_cast}
类型转换四种之一。C语言的类型转换直接在变量前加上类型即可,C++的类型转换有四种
static\_cast,dynamic\_cast,reinterpret\_cast,const\_cast。

dynamic\_cast只适合多态类之间的转换,可以通过这种方式判断一个当前的实体是否是可以转换的,
或者确认当前实体是否是某个类的子类。

编译器是如何知道不能转换的呢?因为在运行的时候编译器存储了类运行时的信息,RTTI。这是会增加开销,

\subsection{结构化绑定}
作者提到了常用的接受多个返回值的方法。用tuple接收:
\begin{lstlisting}
	std::tuple<std::string, int> CreatePerson()
{
    return {"LY", 24};
}

\end{lstlisting}
这个tuple接收两个返回值,一个string,一个int。一种是用get取出值:
\begin{lstlisting}
	auto person = CreatePerson();
    std::string& name = std::get<0>(person);
    int age = std::get<1>(person);
\end{lstlisting}
另一种是tie,相比来说tie更整洁
\begin{lstlisting}
	std::string name;
    int age;
    std::tie(name, age) = CreatePerson();
\end{lstlisting}
如果使用结构化绑定,不需要先定义两个接收变量的类型,直接使用即可:
\begin{lstlisting}
	auto[name, age] = CreatePerson();
\end{lstlisting}
不过这个特性是17后生效。现在22年了大家基本都是大于17版本肯定生效的。

后面两章介绍17新特性的,先跳过。

\subsection{如何能让C++字符串更快?}
前面有提到过,尽量在栈内分配,避免在堆上分配。string就是在堆上分配的,但是我阅读了一下
string的源码,如果是在16字节内,他就是在换存池里面分配,也就是栈,如果是大于16字节那就是
在堆上分配。事实上这也是很自然的,毕竟栈内存在编译后就确定了,如果是动态变量需要很大的内存
空间,栈是满足不了的。

\begin{lstlisting}
	std::string name = "hajksdhjkkqwey";
    std::string first = name.substr(0, 3);
    std::string second = name.substr(4, 9);
\end{lstlisting}
这三步每一步都会分配内存,一共在堆上分配了三个内存。我们并不希望他分配内存,只需要能access
到这个name就可以。

string\_view可以达成,string\_view只是记录了当前string的指针和偏移量,也就是说他只是string
的一个指针类型,相比string需要分配更大的空间,string\_view明显效率更高。
\begin{lstlisting}
	std::string_view first(name.c_str(), 3);
    std::string_view second(name.c_str(), 4);
\end{lstlisting}
c\_str返回一个指向当前字符串的指针,比如char* c = a.c\_str(),本身就是用来兼容c的。
这样就只需要分配一次了。如果要把剩下的这个去除,只需要把string修改即可:
\begin{lstlisting}
	const char* name = "hajksdhjkkqwey";
    // std::string first = name.substr(0, 3);
    // std::string second = name.substr(4, 9);

    std::string_view first(name, 3);
    std::string_view second(name, 4);
\end{lstlisting}
这样一次分配都没有。

\subsection{单例模式}
这应该是常见的一种设计模式。当我们想让当前数据共享于所有的类或者结构体的时候,并且我们希望
保证数据的一致性,单例模式就非常有用。比如渲染器就是一种全局通用的单例,不会因为不同物体
就采用不同的渲染器。当然渲染也分很多种,有体渲染,也有面渲染,但他们都只需要创建一个就行。

这个比较简单,基本上面试笔试都要写。主要就是满足对于每一次调用都只能共用一个实例就行。
创建
\begin{lstlisting}
class Singleton
{
    public:
        Singleton(const Singleton&) = delete;

        static Singleton& Get()
        {
            return s_Instance;
        }
    private:
        Singleton() {}
        static Singleton s_Instance;
};

\end{lstlisting}
构造函数放在private是不允许显示调用,public中还需要静止复制,在private初始化然后
public写个窗口供调用就行。

\subsection{小字符优化}
小字符优化也叫SSO,字符占用字节如果小于16字节,那么会在栈的缓冲区分配内存,如果超过了,
就会堆上返回。这里的源码不一样,要找if CXX14的源码看,他看的是旧版源码,新的用的construct
分配。

在xstring源码可以看到
\begin{lstlisting}
	if (_Count < _BUF_SIZE) {
		_My_data._Mysize = _Count;
		_My_data._Myres  = _BUF_SIZE - 1;
		if constexpr (_Strat == _Construct_strategy::_From_char) {
			_Traits::assign(_My_data._Bx._Buf, _Count, _Arg);
		} else if constexpr (_Strat == _Construct_strategy::_From_ptr) {
			_Traits::move(_My_data._Bx._Buf, _Arg, _Count);
		} else { // _Strat == _Construct_strategy::_From_string
\end{lstlisting}
小于特定的buf\_size的时候不会分配堆内存。

\subsection{跟踪内存分配}
主要还是为了性能服务,在写安全协议的时候也用到,不过是借助了其他语言工具。

就是自己写一个能记录内存分配的类或者工具,感觉还是用给定好的包或者库好用。

\subsection{左值和右值}
最简单的一个例子
\begin{lstlisting}
	int i = 10;
\end{lstlisting}
左值一般在左边,右值一般在右边。但这种规则并不总是适用,比如int a = b这就不适合,两个都是左值。

\begin{outline}
	\1 左值一般来说是可寻址,有一定内存指向的变量。右值一般是临时变量。
	\1 赋值给左值是可以的,给右值不行,而左值是要非const可修改的。
	\1 左值引用,只能是给左值。
	\1 const左值引用可以给右值,const int\& a = 10。实际上是创建了一个临时变量。
	\textbf{所以const左值引用是兼容左值和右值的。这也是为什么string的调用通常使用const的原因,
	因为兼容左值和右值。}
\end{outline}

\begin{lstlisting}
	std::string a = "sjkdf";
    std::string b = "aksla";
    std::string c = a + b;
    a + b;
\end{lstlisting}
c是左值,a+b是右值。因为在c = a + b的过程中调用了string的构造函数,而a + b没有,就是一个右值。

\subsection{参数计算顺序}
跳过了,更考试题一样,这节的内容谁写进工程代码里面我砍谁。

\subsection{移动语义}
为什么需要移动语义?
\begin{itemize}
	\item 通常我们只是希望创建出一个临时变量给构造函数,如果在主进程创建一个变量然后传到
	构造函数,然后构造函数会复制这个变量。
\end{itemize}
一个简单的例子
\begin{lstlisting}
	class String {
public:
    String() = default;
    String(const char* string) {
        printf("Created!\n");
        m_Size = strlen(string);
        m_Data = new char[m_Size];
        memcpy(m_Data, string, m_Size);
    }
 
    String(const String& other) {
        printf("Copied!\n");
        m_Size = other.m_Size;
        m_Data = new char[m_Size];
        memcpy(m_Data, other.m_Data, m_Size);
    }
 
    ~String() {
        delete[] m_Data;
    }
 
    void Print() {
        for (uint32_t i = 0; i < m_Size; ++i)
            printf("%c", m_Data[i]);
 
        printf("\n");
    }
private:
    char* m_Data;
    uint32_t m_Size;
};
 
class Entity {
public:
    Entity(const String& name)
        : m_Name(name) {}
    void PrintName() {
        m_Name.Print();
    }
private:
    String m_Name;
};
\end{lstlisting}
创建变量entity的时候,传入entity(string("a")),首先创建string,然后在构造函数又复制一次。
我们可以强制他传右值,把string的构造函数去掉。增加
\begin{lstlisting}
    String(String&& other) {
        printf("Moved!\n");
        m_Size = other.m_Size;
        m_Data = other.m_Data;
        other.m_Data = nullptr;
        other.m_Size = 0;
    }
\end{lstlisting}
entity构造函数也改成只接收右值
\begin{lstlisting}
    Entity(String&& name)
        : m_Name(name) {}
\end{lstlisting}
但实际上还是没能解决问题,因为右值传进来的时候就已经退化了,在entity的构造函数传递name的
时候就变成了左值,这个时候用std::move即可。\textbf{move实际上是移动资源,有时候需要额外
操作防止内存泄漏。}所以通常需要首先删除当前变量资源,然后赋值,然后删除赋值变量的资源。

\section{C++小游戏}
\url{https://www.bilibili.com/video/BV14z4y1r7wX/?spm_id_from=333.999.0.0&vd_source=a059a118f33728f79abd79e02f8f72d4}
的教程,自己的理解以及一些改进。

\subsection{俄罗斯方块}
创建方块,这里一共7个,作者用的wstring,wstring每一个字符占用2个字节,我看他做的
都是字母,感觉string应该也行。
\begin{lstlisting}
	std::wstring tetromino[7];
\end{lstlisting}
然后挨个挨个赋值就行。

\subsubsection{坐标}
坐标这个问题比较重要,他涉及到碰撞检测,移动,显示等的问题。正常的坐标在左上角开始,xy两个方向
0开始增加:
\begin{lstlisting}
/*
0 1 2 3 4 5
0
1
2
3
4
5
*/
\end{lstlisting}
作者用一维数组表示,每一个小格子用一个index表示,从左上角开始0一直编号下去
\begin{lstlisting}
/*
0 1 2 3 4 5
6 7 8 9 10 11
*/
\end{lstlisting}
对于在(x, y)的点,用一维数组表示就是index = y * w + x。当他顺时针旋转90度
index = 12 + y - (x * h),当顺时针旋转180度index = 15 - (y * w) - x,当顺时针
旋转270度index = (y * w) + (h - x - 1)

\subsubsection{前期工作准备}
没有写过win,这些API看的头疼,只能一个个查了。首先区域分为三个区域
\begin{itemize}
	\item 旋转区域:这个区域是方块按照玩家指示旋转的,上面提的顺时针旋转这些就是。
	   \subitem 这个区域作者没有写出来,只是约定俗成了就是4x4
	我还以为是整个屏幕坐标的,所以那四条公式我都写成w和h。但实际上就是再4x4的格子进行旋转。
	\item 玩家区域:也就是游戏区域,一整个窗口不会都是游戏。
	   \subitem pField就是玩家区域,注意这个pField只是会记录固定的方块,也就是说你这个方块
	   如果是在下落过程中,是不会记录的。只有落下了才会记录。因为pField实际上是就是游戏环境,
	   而下落的方块不是游戏环境,是玩家。就好像迷宫一样,人是玩家,迷宫就是这个pField。
	\item 窗口区域:显示给用户看的整个区域。
	   \subitem screen,需要把pField移动到screen才能显示。
\end{itemize}
玩家控制区域的建立
\begin{lstlisting}
	pField = new unsigned char[nFieldWidth * nFieldWidth];
    for (int x = 0; x < nFieldWidth; x++)
        for (int y = 0; y < nFieldHeight; y++)
            // 边界设置为9否则都设置为0
            pField[y * nFieldWidth + x] = (x == 0 || x == nFieldWidth - 1 || y == nFieldHeight - 1) ? 9 : 0;
    
\end{lstlisting}
0的地方是可以进行移动,9的地方是边界。窗口的创建需要了解几个API:
\begin{outline}
	\1 CreateConsoleScreenBuffer:文档地址\url{https://learn.microsoft.com/zh-cn/windows/console/createconsolescreenbuffer}。
		\2 作用的创建控制台缓冲区。第一个参数是访问权限,代码里面是read和write都可以。第二个参数
		表示缓冲区是否共享,意思就是这个缓冲区能不能被其他线程access,肯定是不能,所以设置为0。
		第三个参数表示句柄能不能让子线程继承,NULL就是不能了。第四个参数是类型,这个类型没得选。
		最后一个参数是否包含这个数据,NULL表示保护。
		\2 返回句柄,表示创建了一个能被主进程读写,不能被其他进程共享,不被继承并且数据被保护类型指定的缓冲区。
    \1 SetConsoleActiveScreenBuffer:将指定缓冲区设置为屏幕控制缓冲区。emmm人家文档好像不建议用这个,写完再看看。
	\1 WriteConsoleOutputCharacter:将字符写入到缓冲区。
		\2 第一个参数表示写入的缓冲区,要求access必须是可写的。第二个参数要写入的字符。
		第三个参数是写入的字符数,第四个参数写入的字符坐标,默认00就是在左上角,如果
		想要在中间那可以在这调整,第五个参数表示一个指针,表示实际写入的字符数,作者
		用DWORD接收,这是MFC的数据类型,32位无符号数。
\end{outline}
所以他是先创建了一个缓冲区,然后把这个缓冲区设置成屏幕的缓冲区,让他显示在屏幕上,每一次
游戏更新,把更新的数据写进缓冲区就行。作者的缓冲区用wchar表示,看代码实例里面
大多数都是用一个共用缓冲区的联合体表示。

在这里运行的时候出现一个问题,在对wchar\_t进行内存分配的时候,居然出现分配错误,
一开始以为是os内存不够,但是后面看了一下是完全够的。后面看了一下是屏幕缓冲区的问题,调大点
就没事了。

然后在测试的时候发现,屏幕显示有问题,这个窗口大小太大了。MoveWindow,SetWindow,Setbuffer
这些都没有效果,返回值显示都成功了,但就是没效果。后面上外网看了一下,他这个是因为visual 
studio和clion都强制设置了窗口大小,对没错,为了解决这个问题我还下了clion。后面我在dev c++
上用MoveWindow,SetWindow,Setbuffer方法是可以的。说明这个问题是和编译器相关。反正只需要手动
调整这个窗口大小就行了。

\subsubsection{碰撞检测}
用0表示空区域。方块的旋转是在一个局部空间内进行,先在局部空间内进行旋转,然后再转移
到全局坐标上去。
\begin{lstlisting}
if (nPosX + px >= 0 && nPosX + px < nFieldWidth)
	if (nPosY + py >= 0 && nPosY + py < nFieldHeight)
	{
		if (tetromino[nTetromino][pi] == L'X' && pField[fi] != 0)
		{
			return false;
		}
	}
\end{lstlisting}
碰撞检测主要就是判断当前点移动之后,会不会碰到其他的点。如果是空白区域那就是0,否则就是不是。

\subsubsection{控制移动}
检测按键
\begin{lstlisting}
	for (int i = 0; i < 4; i++)
	{   
		// 检查最高位是不是1
		// 虚拟键码里面上,左,右分别对应38,37,39,十六进制对应27 25 28,还有一个Z用来变换
		bKey[i] = (0x8000 & GetAsyncKeyState(static_cast<unsigned char>("\x27\x25\x28Z"[i]))) != 0;
	}
\end{lstlisting}
GetAsyncKeyState主要是检测按键,x27表示27是一个十六进制的。用虚拟键码表示按键。
0x8000做与或运算,和最高位对比,如果最高位是1说明就按下了,接下来就是移动了。
\begin{lstlisting}
	nCurrentX += (bKey[1] && DoesPieceFit(nCurrentPiece, nCurrentRotation, nCurrentX - 1, nCurrentY)) ? 1 : 0;
	nCurrentX += (bKey[0] && DoesPieceFit(nCurrentPiece, nCurrentRotation, nCurrentX + 1, nCurrentY)) ? 1 : 0;
	nCurrentY += (bKey[2] && DoesPieceFit(nCurrentPiece, nCurrentRotation, nCurrentX, nCurrentY + 1)) ? 1 : 0;
\end{lstlisting}
旋转需要注意的是如果不加限制,会一直旋转的,因为nCurrentRotation一直都不会是0,所以需要一个
flag来确定是否需要旋转。
\begin{lstlisting}
	if (bKey[3])
	{
		nCurrentRotation += (!bRotateHold && DoesPieceFit(nCurrentPiece, nCurrentRotation + 1, nCurrentX, nCurrentY)) ? 1 : 0;
		bRotateHold = true;
	}
	else
		bRotateHold = false;
\end{lstlisting}
bRotateHold来确定当前是否是继续旋转。不能直接把nCurrentRotation设置成0,因为每一次旋转
都是接着上一次。

\subsubsection{下一块}
首先要添加一个自动下落,这个比较容易,bForceDown表示就行。
\begin{lstlisting}
	if (DoesPieceFit(nCurrentPiece, nCurrentRotation, nCurrentX, nCurrentY + 1))
	// 到底了
   {
	   ++nCurrentY;
   }
\end{lstlisting}
如果能下落,就继续下落,如果下面一个位置是不能下落的,那么就不下落了。既然不下落了,那就
要把这个物体固定在环境中,就是固定在pField中,前面提到过了这个pField实际上是游戏环境,
方块是玩家,方块到底就变成一个环境了。就好像CS人在移动的时候他就是玩家,死了就变成环境的
一部分。
\begin{lstlisting}
for (int px = 0; px < 4; px ++)
	for (int py = 0; py < 4; py++)
	{
		if (tetromino[nCurrentPiece][Rotate(px, py, nCurrentRotation, 4, 4)] == L'X')
		{
			pField[(nCurrentY + py) * nFieldWidth + (nCurrentX + px)] = nCurrentPiece + 1;
		}
	}
\end{lstlisting}
这里要注意的是tetromino[nCurrentPiece][Rotate(px, py, nCurrentRotation, 4, 4)]表示当前
物体的状态,这里为什么要有一个Rotate呢?因为这个控制旋转的逻辑不是subsequent的,并不是说
点击一次旋转,0到90度,再点击一次90度到180度。而是点击一次0到90度,再点击一次0到180度。

这段代码相当于是把当前方块给画出来了。其实这个代码重复了很多遍,可以写成一个函数。设置
下一块的话重新设置就好了
\begin{lstlisting}
	nCurrentX = nFieldWidth / 2;
	nCurrentY = 0;
	nCurrentPiece = rand() % 7;
	nCurrentRotation = 0;
\end{lstlisting}

\subsubsection{删除}
主要是指删除某一行。这个检测是否一行都有方块并不是什么时候都要检查,肯定是等下落到底了之后
再检查。只需要检查四行就行,因为当前这个方块就占4行,如果超过这个范围就不关你的事了,是之前
或者之后下落的物体要检查的。
\begin{lstlisting}
	for (int py = 0; py < 4; py++)
	{
		if (nCurrentY + py < nFieldHeight - 1) {
			bool bLine = true;
			for (int px = 1; px < nFieldWidth - 1; px++)
			{   
				// 当前这个高度,全部的px都不是0,也就都不是空
				bLine &= (pField[(nCurrentY + py) * nFieldWidth + px]) != 0;
			}
			if (bLine)
			{
				// Remove Line
				for (int px = 1; px < nFieldWidth - 1; px++)
				{
					pField[(nCurrentY + py) * nFieldWidth + px] = 8;
				}
				// 要删除的行
				vLines.push_back(nCurrentY + py);
			}
		}
	}
\end{lstlisting}
当前整行都不是空,那么就把这行放进vLines里面,等会删掉他。
\begin{lstlisting}
	for (auto& v : vLines)
	{
		for (int px = 1; px < nFieldWidth - 1; px++)
		{
			for (int py = v; py > 0; py--)
			{
				pField[py * nFieldWidth + px] = pField[(py - 1) * nFieldWidth + px];
			}
			pField[px] = 0;
		}
	}
	vLines.clear();
\end{lstlisting}
pField[px] = 0;这一句我一开始不知道是干什么,后面想了一下,我觉得应该是处理特殊情况的。
当你所以方块都填满了整个pField,你这个下移动只会影响1到nFieldHeight-1行,第0行是没有东西
可以给他降下来,没有-1行,所以这里额外需要处理一下。

\subsubsection{渲染}
pField和移动方块是两种不同的物体,一个是环境一个是玩家,自然渲染不一样。
\begin{lstlisting}
	for (int x = 0; x < nFieldWidth; x++)
	for (int y = 0; y < nFieldHeight; y++)
		// 乘上nScreenWidth是因为整个窗口长度就是nScreenWidth,所以需要乘上这一行
		// 边界是#号,其他地方为空格
		screen[(y + 2) * nScreenWidth + (x + 2)] = L" ABCDEFG=#"[pField[y * nFieldWidth + x]];

for (int px = 0; px < 4; px ++)
	for (int py = 0; py < 4; py++)
	{
		if (tetromino[nCurrentPiece][Rotate(px, py, nCurrentRotation, 4, 4)] == L'X')
		{
			screen[(nCurrentY + py + 2) * nScreenWidth + nCurrentX + px + 2] = nCurrentPiece + 65;
		}
	}

\end{lstlisting}
如果想显示什么分数啊什么的,再pField写就行了,如果想增加新的方块,那就要额外渲染。

\subsection{第一人称射击游戏}

\section{CS106B C++ 抽象编程}
\subsection{Welcome}
第一节课主要解决的问题是
\begin{outline}
    \1 Why take CS106B
    \1 What is an abstraction
    \1 What is CS106B
    \1 Why C++
    \1 What's next
\end{outline}

什么是抽象?抽象的定义
\begin{quotation}
    Design that hides the details of how something works while still allowing
     the user to access complex functionality.
\end{quotation}
手机,汽车就是一种抽象,当我们使用汽车手机的时候,也就是access,并不需要了解他们的内部构造,
也就是hide the detail了。另一个例子就是程序语言了,程序语言的抽象核心在于
\begin{quotation}
    Through a simpler interface, user are able to take full advantage of a complex 
    system without needing to know how it works or how it was made.
\end{quotation}



\printbibliography                               
\end{sloppypar}
\end{document}